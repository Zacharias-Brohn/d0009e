\documentclass[a4paper,12pt]{article}
\usepackage[swedish]{babel}
\usepackage[utf8]{inputenc}
\usepackage{amsmath, amsthm, amssymb}
\usepackage[a4paper,includeheadfoot,margin=2.54cm]{geometry}
\usepackage[left]{lineno}
\newtheorem{theorem}{Theorem}
\usepackage{framed}

\newcommand*\patchAmsMathEnvironmentForLineno[1]{%
    \expandafter\let\csname old#1\expandafter\endcsname\csname #1\endcsname
    \expandafter\let\csname oldend#1\expandafter\endcsname\csname end#1\endcsname
    \renewenvironment{#1}%
    {\linenomath\csname old#1\endcsname}%
    {\csname oldend#1\endcsname\endlinenomath}}% 
\newcommand*\patchBothAmsMathEnvironmentsForLineno[1]{%
    \patchAmsMathEnvironmentForLineno{#1}%
    \patchAmsMathEnvironmentForLineno{#1*}}%
\AtBeginDocument{%
    \patchBothAmsMathEnvironmentsForLineno{equation}%
    \patchBothAmsMathEnvironmentsForLineno{align}%
    \patchBothAmsMathEnvironmentsForLineno{flalign}%
    \patchBothAmsMathEnvironmentsForLineno{alignat}%
    \patchBothAmsMathEnvironmentsForLineno{gather}%
    \patchBothAmsMathEnvironmentsForLineno{multline}%
}
\renewcommand\linenumberfont{\normalfont\bfseries\small}

% -------------------------------------------------------------

\title{Om att skriva tydliga \LaTeX-manus}
%
\author{Håkan Jonsson\thanks{email: \texttt{hj@ltu.se}} \\  
        ~ \\
        Luleå tekniska universitet \\ 
        971 87 Luleå, Sverige}
%          
\date{\today}

\begin{document}

\linenumbers

\maketitle

\begin{abstract}
    \LaTeX-manus ska vara lätta att läsa och förstå. Detta uppnås genom
    logisk gruppering av manusets delar som gör manuset strukturerat. 
\end{abstract}

\section{Introduktion}

Eftersom \LaTeX-manus ska läsas och förstås av inte bara maskiner utan
också männsikor ska de vara strukturerade och välskrivna. Vi säger att
innehållet ska vara \emph{logiskt grupperat och formulerat}. Med detta
menas att kommandon och andra grupperingsinstruktioner skrivs på ett
sätt så en människa lätt kan se vilka delar som hör ihop och vad som
påverkas/är del av/innesluts av vad.

Att skriva \LaTeX-manus är som att programmera, fast med den
skillnaden att resultatet blir ett dokument och inte ett körbart
datorprogram. Precis som det i detta dokument radas upp ett antal
skrivregler för \LaTeX-manus finns i regel motsvarande regler då man
programmerar t~ex python, Java eller C. En sån regelsamling kallas
\emph{kodningsstandard}. 

I avsnitt~\ref{sect:meningar} nedan beskrivs hur matematiska symboler,
formler, uttryck mm ska gå att läsa som vanlig text och tillsammans
med alla annan text. Sen förklaras kort hur du ska skriva dina
\LaTeX-manus i avsnitt~\ref{sect:skrivregler}. Men först går vi i
nästa avsnitt igenom några enkla sätt, här kallade ''verktyg'', att
förbättra ett manus läsbarhet på. 

\section{Verktyg}

Vi använder 4 olika ''verktyg'' för att göra ett manus lättläst och
tydligt. De är \emph{indentering}, \emph{radbrytning}, \emph{justering
i horisontalled} och \emph{separering i vertikalled}. I det följande
beskrivs vart och ett i ordning.  

\subsection*{Verktyg 1: Indentering}

Detta innebär att kod och text startar en bit in på raderna.
Manusdelar inne i en omgivning ska konsekvent indenteras in
ytterligare i förhållande till hur omgivningens \texttt{begin} och
\texttt{end} indenterats. Hur mycket anges i mellanslag och vanliga
antal är 2, 4 och 8. Om vi t~ex indenterar med 4 mellanslag så skriver
vi om koden  
%
\begin{verbatim}
\begin{theorem}
\begin{displaymath}
f(n) = 
\begin{cases}
1, & \text{om $n=0$ och} \\
nf(n-1) & \text{annars.}
\end{cases}
\end{displaymath}
\end{theorem}
\end{verbatim}
%
till~(\verb*! ! markerar här ett mellanslag): 
%
\begin{verbatim*}
\begin{theorem}
    \begin{displaymath}
        f(n) = 
        \begin{cases}
            1, & \text{om $n=0$ och} \\
            nf(n-1) & \text{annars.}
        \end{cases}
    \end{displaymath}
\end{theorem}
\end{verbatim*}
%
Notera hur kod som är inuti annan kod indenteras ytterligare 4
mellanslag. I den senare kodversionen ovan är t~ex 
\begin{itemize}
  \item \dots raden \verb! 1 & \text{om $n=0$ och} \\! indenterad 4
    mellanslag mer än

  \item \dots raden med \verb!\begin{cases}! som är lika mycket indenterad som

  \item \dots raden med \verb!f(n) =! som är indenterad 4 mellanslag mer än

  \item \dots raden med \verb!\begin{displaymath}! som i sin tur är
    indenterad 4 mellanslag mer än

  \item raden med \verb!\begin{theorem}! som har indentering 0.
\end{itemize}
I de allra flesta fall har indenteringen ingen betydelse för hur det
färdiga dokumentet kommer att se ut. Båda koderna ovan ger
%    
\begin{theorem}
    \begin{displaymath}
        f(n) = 
        \begin{cases}
            1, & \text{om $n=0$ och} \\
            nf(n-1) & \text{annars.}
        \end{cases}
    \end{displaymath}
\end{theorem}
%   
\noindent Men den indenterade versionen är mycket lättare att förstå.

\subsection*{Verktyg 2: Radbrytning}

Vi fyller vanligen raderna med kod och text. Om det bättrar på läsbarheten
är det dock befogat att radbryta texten inne på en rad. Detta betyder
att man inte använder hela raden och är vanligt i långa matematiska
uttryck, som annars kan vara svåra att begripa. T~ex är det inte så
enkelt att omedelbart se att koden
%    
\begin{verbatim}
\begin{align*}
    (x+h)^2-x^2&=x^2+2xh+h^2-x^2\\&=2xh+h^2\\&=h(2x+h)
\end{align*}
\end{verbatim}
%
blir
%    
\begin{align*}
    (x+h)^2-x^2&=x^2+2xh+h^2-x^2\\&=2xh+h^2\\&=h(2x+h)
\end{align*}
%
i det färdiga dokumentet. Betydligt mera lättläst är det att
istället skriva koden som
%
\begin{verbatim}
\begin{align*}
    (x+h)^2-x^2&=x^2+2xh+h^2-x^2\\
    &=2xh+h^2\\
    &=h(2x+h).
\end{align*}
\end{verbatim}

\subsection*{Verktyg 3: Justering i horisontalled}

Detta innebär att stoppa in mellanslag inne på rader för att 
1) dels tydligt separera innehållsdelar åt,
2) dels justera likartat innehåll på efter varandra följande
   rader mot varandra horisontellt. 
Koden i radbrytningsexemplet ovan blir t~ex ännu lättare att
läsa och förstå om vi glesar ut och justerar den så här:
%
\begin{verbatim}
\begin{align*}
    (x + h)^2 - x^2 &= x^2 + 2xh + h^2 - x^2 \\
                    &= 2xh + h^2             \\
                    &= h(2x + h).
\end{align*}
\end{verbatim}
%
Här har vi stoppat in mellanslag runt operatorer, så utryckens
termer klarare ska framgå. Vi har också skjutit in de sista
två raderna så alla tre \verb!&=! hamnar under varandra. Samma
förbättring kan göras i t~ex tabeller. Figur~\ref{fig:f1} visar ett
exempel\footnote{Tabellen visar alla som vunnit Formel~1-VM minst 4
gånger till år 2021.}.
%
\begin{figure}[h]
    \centering
    \begin{tabular}{|l|c|l|}
        \hline \hline
        Förare & Vinster & Säsonger \\
        \hline \hline      
        Michael Schumacher & 7 & 1994--1995, 2000--2004 \\
        \hline
        Lewis Hamilton & 7 & 2008, 2014--2015, 2017--2020 \\
        \hline
        Juan Manuel Fangio & 5 & 1951, 1954--1957 \\
        \hline
        Alain Prost & 4 & 1985--1986, 1989, 1993 \\
        \hline
        Sebastian Vettel & 4 & 2010--2013 \\ 
        \hline \hline
    \end{tabular}
    \caption{Vinnare av förarmästerskapet i Formel 1}
    \label{fig:f1}
\end{figure}

Det går att skapa en sån här tabell på många sätt i \LaTeX. Här är ett
sätt som använder \texttt{tabular}-omgivningen:
%
\begin{verbatim}
\begin{tabular}{|l|c|l|} 
\hline \hline Förare & Vinster & Säsonger \\ \hline \hline Michael
Schumacher & 7 & 1994--1995, 2000--2004 \\ \hline Lewis Hamilton &
7 & 2008, 2014--2015, 2017--2020 \\ \hline Juan Manuel Fangio & 5 &
1951, 1954--1957 \\ \hline Alain Prost & 4 & 1985--1986, 1989, 1993 \\
\hline Sebastian Vettel & 4 & 2010---2013 \\ \hline \hline
\end{tabular}
\end{verbatim}
%
Dessa 7 kodrader är väldigt svåra att tyda. Om man benar upp raderna och
använder verktyg~1 och 2, dvs både \emph{indenterar} och \emph{radbryter}
de uppbenade raderna, får vi istället det betydligt mera lättbegripliga 
%
\begin{verbatim}
\begin{tabular}{|l|c|l|}
    \hline \hline
    Förare & Vinster & Säsonger \\
    \hline \hline      
    Michael Schumacher & 7 & 1994--1995, 2000--2004 \\
    \hline
    Lewis Hamilton & 7 & 2008, 2014--2015, 2017--2020 \\
    \hline
    Juan Manuel Fangio & 5 & 1951, 1954--1957 \\
    \hline
    Alain Prost & 4 & 1985--1986, 1989, 1993 \\
    \hline
    Sebastian Vettel & 4 & 2010---2013 \\ 
    \hline \hline
\end{tabular}
\end{verbatim}
%    
Även denna kods läsbarhet går dock att förbättra genom att amvända
verktyget \emph{justering i horisontalled}.
%
\begin{verbatim}
\begin{tabular}{|l|c|l|}
    \hline \hline
    Förare             & Vinster & Säsonger                     \\
    \hline \hline      
    Michael Schumacher &    7    & 1994--1995, 2000--2004       \\
    \hline
    Lewis Hamilton     &    7    & 2008, 2014--2015, 2017--2020 \\
    \hline
    Juan Manuel Fangio &    5    & 1951, 1954--1957             \\
    \hline
    Alain Prost        &    4    & 1985--1986, 1989, 1993       \\
    \hline
    Sebastian Vettel   &    4    & 2010---2013                  \\ 
    \hline \hline
\end{tabular}
\end{verbatim}    

\subsection*{Verktyg 4: Separering i vertikalled}

Denna typ av separering görs med tomrader och procenttecken. Om det
passar med ett nytt stycke, kan vi separera med en eller flera
tonmrader efter varandra; tomrader översätts ju nytt stycke i det
färdiga pdf-dokumentet. Där nytt stycke inte passar kan vi istället
lägga in rader som endast innehåller ett procenttecken
(\verb!%!). Allt på en rad efter ett procenttecken, inkludive
procenttecknet, ignoreras av \LaTeX-systemet. 

Manuset till detta dokument innehåller många exempel på separering
i vertikalled med både tomrad och ''procentrader''. Notera särskilt
hur alla rubrikrader, möjligen inklusive etikettrader, omges av tomrader. 

\section{Allt är meningar}
\label{sect:meningar}

Matematiska uttryck bara är \emph{förkortade skrivsätt} för vanlig
text, som är nedskrivna tankar. Att vi överhuvudtaget använder
matematisk notation och matematiska symboler är för att det är lättare
att vara \emph{entydlig} med sån jämfört med vanlig text. Matematiken
tillför inget annat. 

Eftersom matematisk text endast är komprimerade versioner av vanlig
text så ska det alltid gå att läsa de matematiska delarna av en text som att
den ingår i hela textflödet. Alltså, de ekvationer och formler som
finns ska gå att ersätta med vad de egentligen betyder och allt ska sen
gå att läsa. Här har vi ett litet exempel från elektrotekniken:
%
\begin{framed}
    \noindent Enligt elkretsteori är spänningen
%
    \begin{equation}
        \label{eq:1}  
        U = RI,        
    \end{equation}  
%
    där $R$ är resistansen och $I$ strömmen. 
\end{framed}
%
Detta går lätt att läsa:
%
\begin{quote}
    ''Enligt elkretsteori är spänningen $U$ lika med $R$
    gånger $I$, där $R$ är resistansen och $I$ strömmen.''
\end{quote}
%
Och sen kan vi fortsätta med t~ex att
skriva 
%
\begin{framed}
  \noindent För effekten, $P$, gäller att $P = IU$.
  Ekvation~\ref{eq:1} insatt i effektsambandet ger då att  
%
    \begin{equation}
        P = RI^2,
    \end{equation}
%
    ett alternativt effektsamband som beror av resistans och ström
    snarare än spänning och ström.
\end{framed}

Läs även det ovan högt för dig själv. Notera kommatecknet efter den
sista formeln eftersom meningen fortsätter ända fram till punkt. 

\section{Några råd om hur  \LaTeX-manus bör skrivas}
\label{sect:skrivregler}

Manus skrivs inte bara för LaTeX-systemet utan även för dig själv och
andra människor. Ett konsekvent utformat och i övrigt välskrivet manus
är lätt att förstå även långt senare. Det främjar även samarbete.

I det riktigt bra manuset används genomgående de fyra verktygen 
\begin{itemize}
  \item indentering,
  
  \item radbrytning,

  \item justering i horisontalled och

  \item separering i vertikalled
\end{itemize}
%
som tidigare introducerats. Viktigt är då att användningen av
verktygen är konsekvent genom hela manuset. T~ex är indenteringen
densamma överallt och oberoende av indenteringsdjup. Man kan
sammanfatta det med att samma skrivregler ska vara tillämpade genom
hela manuset och att det i manusets hela kod ska råda ''ordning och
reda''. 

\end{document}