\documentclass[a4paper,12pt]{article}
\usepackage[swedish]{babel}
\usepackage[utf8]{inputenc}
\usepackage{amsmath, amsthm, amssymb}
\usepackage[a4paper,includeheadfoot,margin=2.54cm]{geometry}
\usepackage{enumerate}
\renewcommand{\familydefault}{\sfdefault}

\title{D0015E Datateknik och ingenjörsvetenskap \\
       Uppgift i moment L om yrkesrollen}
%
\author{Zacharias Brohn}
%
\date{\today}

\begin{document}

\maketitle

% Kom ihåg att ändra författarnamn ovan till ditt eget namn Ändra inte någon
% avsnittsrubrik

\section*{Svar på uppgift 1}

% Skriv svaren ett i taget nedan. 

\begin{enumerate}
  %
  \item En jämställd och könsblandad arbetsplats visar på att företaget är
  flexibel och hållbar, det är enklare att både anställa och behålla
  intellektuell arbetskraft oberoende kön. Det första som behövs för detta är
  som vi kallar 'könsblandad' arbetsplats, alltså att könsuppdelningen inom
  företaget ska likna 50/50, hälften män och hälften kvinnor. Det i sig självt
  skapar inte en jämställd arbetsplats däremot, utan insatser att minska
  enkönsuppdelning inom företaget så som mans- eller kvinnodominerade grupper
  och relationer.
  %
  \item Nyttjar till en bekväm och inkluderande arbetsmiljö där kvinnor och män
  kan arbeta tillsammans utan fördomar vilket leder till en starkare
  arbetskraft för företaget. Det nämndes kort i tidigare svar om enkönade
  grupper men ytterligare insaster som att förändra yrket för att passa olika
  typer av människor, kanske inte endast för att passa 'kvinnor och män', där
  relationer mellan könen förbättras för att förminska stereotyper och normer.
  %
  \item Förbättrar lojalitet och arbetslust hos anställda, grundproblemet här
  är att minoriteten blir tilldelade stereotypiska roller samt att det kan
  finnas behov av att överprestera för att kompensera eller 'bevisa' att hen
  passar in, vara med i gänget. Detta är då inte ett enskilt problem i sig och
  det betyder då att det krävs insatser i olika områden, många av de har jag
  redan nämnt i tidigare svar, men för att utveckla vidare kan det behövas
  insatser för att motverka samt förebygga motståndet mot förändring och
  jämställdhet, där företag kan medvetet eller omedvetet hindra jämställdhet på
  grund av olika anledningar så som otillräcklig kompetens kring området, då
  blir det viktigt att informera och lägga grunden för hur det går att uppnå
  jämställdhet.
  %
\end{enumerate}

\section*{Svar på uppgift 2}

% Skriv ditt svar på uppgiften här.

\begin{enumerate}
  %
  \item Motverka normer om hur kvinnor och män 'bör' vara eller bete sig,
  vilket är något som har blivit bättre med åren men även idag finns det
  könsspecifika sociala stereotyper som beskriver vilka typer av yrken är för
  män och kvinnor, alltså skapa en miljö där det inte finns förväntade svar
  från könen.
  %
  \item Minska fysisk uppdelning beroende på kön inom företaget vilket bidrar
  till att minska skillnaden på förutsättningar mellan könen.
  %
  \item Det är lätt att fokusera insatser och hjälp till minoriteten i
  företaget men det är också viktigt att tillämpa insatser på majoriteten 
  %
\end{enumerate}
  
\section*{Svar på uppgift 3}

% Skriv ditt svar på uppgiften här.

% Om referens till kompendiet behövs så använd denna: \cite{komp}

\section*{Svar på uppgift 4}

% Skriv ditt svar på uppgiften här.

\section*{Svar på uppgift 5}

% Skriv ditt svar på uppgiften här.

\begin{thebibliography}{99}
  \bibitem{komp} Sven Ove Hansson. \emph{Teknik och etik}.  Kompendium, 2009.
    \\
    URL: \verb|https://people.kth.se/~soh/tekniketik.pdf| \\
    Läst 2021-10-08. 
  \end{thebibliography}
  %
\end{document}
