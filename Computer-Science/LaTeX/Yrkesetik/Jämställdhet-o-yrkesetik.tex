\documentclass[a4paper,12pt]{article}
\usepackage[swedish]{babel}
\usepackage[utf8]{inputenc}
\usepackage{amsmath, amsthm, amssymb}
\usepackage[a4paper,includeheadfoot,margin=2.54cm]{geometry}
\usepackage{enumerate}
\renewcommand{\familydefault}{\sfdefault}
%
\title{D0015E Datateknik och ingenjörsvetenskap \\
       Uppgift i moment L om yrkesrollen}
%
\author{Zacharias Brohn}
%
\date{\today}
%
\begin{document}
%
\maketitle
\section*{Svar på uppgift 1}
\begin{enumerate}
    \item En jämställd och könsblandad arbetsplats visar på att företaget är
          flexibel och hållbar, det är enklare att både anställa och behålla
          intellektuell arbetskraft oberoende kön. Det första som behövs för detta är
          som vi kallar 'könsblandad' arbetsplats, alltså att könsuppdelningen inom
          företaget ska likna 50/50, hälften män och hälften kvinnor. Det i sig
          självt skapar inte en jämställd arbetsplats däremot, utan insatser att
          minska enkönsuppdelning inom företaget så som mans- eller kvinnodominerade
          grupper och relationer.
    %
    \item Nyttjar till en bekväm och inkluderande arbetsmiljö där kvinnor och
          män kan arbeta tillsammans utan fördomar vilket leder till en starkare
          arbetskraft för företaget. Det nämndes kort i tidigare svar om enkönade
          grupper men ytterligare insaster som att förändra yrket för att passa olika
          typer av människor, kanske inte endast för att passa 'kvinnor och män', där
          relationer mellan könen förbättras för att förminska stereotyper och
          normer.
    %
    \item Förbättrar lojalitet och arbetslust hos anställda, grundproblemet här
          är att minoriteten blir tilldelade stereotypiska roller samt att det kan
          finnas behov av att överprestera för att kompensera eller 'bevisa' att hen
          passar in, vara med i gänget. Detta är då inte ett enskilt problem i sig
          och det betyder då att det krävs insatser i olika områden, många av de har
          jag redan nämnt i tidigare svar, men för att utveckla vidare kan det
          behövas insatser för att motverka samt förebygga motståndet mot förändring
          och jämställdhet, där företag kan medvetet eller omedvetet hindra
          jämställdhet på grund av olika anledningar så som otillräcklig kompetens
          kring området, då blir det viktigt att informera och lägga grunden för hur
          det går att uppnå jämställdhet.
    %
\end{enumerate}
%
\section*{Svar på uppgift 2}
\begin{enumerate}
    %
    \item Motverka normer om hur kvinnor och män 'bör' vara eller bete sig,
          vilket är något som har blivit bättre med åren men även idag finns det
          könsspecifika sociala stereotyper som beskriver vilka typer av yrken är för
          män och kvinnor, alltså skapa en miljö där det inte finns förväntade svar
          från könen.
    %
    \item Minska fysisk uppdelning beroende på kön inom företaget vilket bidrar
          till att minska skillnaden på förutsättningar mellan könen.
    %
    \item Det är lätt att fokusera insatser och hjälp till minoriteten i
          företaget men det är också viktigt att tillämpa insatser på majoriteten
    %
\end{enumerate}
%
\section*{Svar på uppgift 3}
\begin{enumerate}
    \item \textbf{Ansvarskänsla} - Ingenjören har ett särskilt ansvar för att
          tekniska lösningar inte ska orsaka skada för människor, miljö eller
          samhället. Detta innebär att ingenjören bör agera med omsorg och
          noggrannhet för att undvika risker och negativa konsekvenser av tekniken.
    %
    \item \textbf{Noggrannhet} - Ingenjörer måste vara extremt noggranna i sitt
          arbete eftersom tekniska misstag kan få allvarliga konsekvenser. Det gäller
          att säkerställa att tekniska lösningar är pålitliga och säkra.
    %
    \item \textbf{Ärlighet och integritet} - Precis som läkare och advokater
          förväntas ingenjörer hålla sig till sanningen och arbeta på ett sätt som
          bygger förtroende. Det handlar bland annat om att vara ärlig när problem
          uppstår och om att ta ansvar för eventuella misstag.
    %
    \item \textbf{Problemlösningsförmåga} - Ingenjörer förväntas vara skickliga
          på att lösa komplexa problem på ett systematiskt och vetenskapligt sätt.
          Det krävs också ett analytiskt tänkesätt för att kunna väga olika lösningar
          och deras konsekvenser.
    %
    \item \textbf{Samarbetsförmåga} - Eftersom tekniska projekt ofta involverar
          många aktörer, förväntas ingenjören kunna samarbeta med kollegor och andra
          yrkesgrupper för att nå gemensamma mål. Den sociala aspekten av
          ingenjörsarbetet är viktig för framgång.
    %
    \item \textbf{jälvständighet} - En ingenjör förväntas kunna fatta egna,
          självständiga beslut inom sitt ansvarsområde. Detta innebär att man inte
          bara följer givna instruktioner utan också kan göra egna bedömningar och ta
          initiativ när det behövs.
    %
    \item \textbf{Framåtanda och innovationsförmåga} - Ingenjörer bör också
          sträva efter att förbättra tekniken och utveckla nya lösningar. Detta
          kräver en vilja att vara kreativ och tänka framåt för att lösa framtida
          utmaningar och förbättra människors liv.
    %
    \item \textbf{Lojalitet mot samhällets bästa} - Ingenjörer har ett ansvar
          att ta hänsyn till samhällets behov och säkerhet, vilket innebär att de
          inte enbart kan vara lojala mot arbetsgivare eller kunder. Detta kan
          innebära att vara beredd att slå larm eller stå upp mot omoraliska beslut
          om tekniken riskerar att skada människor eller miljö.
\end{enumerate}
%
\section*{Svar på uppgift 4}
%
\subsection*{A. Varför ingenjörer bör ha legitimation}
\begin{enumerate}
    \item \textbf{Ansvarsfulla yrkesroller} - Precis som läkare och
          sjuksköterskor bär ingenjörer ett stort ansvar för människors säkerhet och
          välbefinnande genom sina tekniska lösningar. Misstag kan leda till
          katastrofala följder som olyckor, miljöförstöring eller till och med
          förlust av liv. Att införa legitimation skulle kunna fungera som en extra
          garanti för att ingenjörer håller en hög professionell standard.
    \item \textbf{Ökad kontroll och ansvar} - En legitimation skulle möjliggöra
          återkallande av rätten att utöva yrket vid grova försummelser eller etiska
          överträdelser, vilket skulle fungera som en tydlig konsekvens för dem som
          inte lever upp till kraven på ansvar och säkerhet. Detta kan främja bättre
          yrkesutövning och en starkare ingenjörsetik.
    \item \textbf{Skydd för allmänheten} - Precis som med läkare och
          sjuksköterskor skulle en legitimation skydda allmänheten från inkompetenta
          eller oetiska ingenjörer. Det skulle också bidra till ett ökat förtroende
          för yrkesrollen och dess aktörer.
\end{enumerate}
%
\subsection*{B. Varför ingenjörer inte bör ha legitimation}
\begin{enumerate}
    \item \textbf{Mångfalden av ingenjörsyrken} - Ingenjörsyrket är mycket
          brett och varierar mellan olika specialiseringar. Att skapa ett enhetligt
          legitimeringssystem för alla ingenjörer skulle vara svårt eftersom det är
          svårt att dra gränser för vad som ingår i ingenjörsarbete och hur olika
          yrkesroller skulle regleras.
    \item \textbf{Komplexa ansvarsfrågor} - Ingenjörers arbete är ofta
          kollektivt och involverar många olika aktörer, vilket gör det svårare att
          peka ut enskilda personer som ansvariga för ett tekniskt misslyckande. Det
          kan vara svårt att avgöra vems legitimation som skulle återkallas om något
          går fel i stora projekt.
    \item \textbf{Administrativ börda} - Att införa ett system för legitimation
          skulle kräva omfattande resurser för både initiala bedömningar och
          uppföljning. Det skulle också skapa byråkratiska hinder för ingenjörer och
          företag, särskilt i en internationell miljö där olika länder har olika
          system för utbildning och certifiering.
\end{enumerate}
%
\subsection*{C. Vad tycker jag?}
Jag tycker inte att det finns behov av en generell legitimation för ingenjörer på samma sätt som
läkare eller sjuksköterskor, och orsaken är att ingenjörsyrket är så brett och
varierande att det skulle vara svårt att upprätthålla ett enhetligt
system. Många ingenjörer arbetar dessutom med teknik
som inte direkt påverkar människors liv och säkerhet på samma sätt som
medicinsk vård gör, vilket sänker ansvaret.
%
Däremot tycker jag att certifieringar bör tillämpas för hög-riskområden och
säkerhetskritiska projekt som broar, kärnkraftverk, medicinteknisk
utrustning, mm. Med det säkerställer man att ingenjörer i dessa specifika
områden håller hög standard utan att införa ett onödigt tungt system för alla
ingenjörer.
%
\section*{Svar på uppgift 5}
Jag ska besvara fråga nr. 7 från \cite{komp} sid. 100. Frågan lyder:
\begin{quote}
    Du arbetar som programmerare på ett konsultföretag. Företaget har gått
    ganska dåligt på sista tiden och har nu till råga på allt blivit stämt av
    en kund som anser att den levererade programvaran inte svarar mot
    specifikationerna. Du får till uppgift att hjälpa ditt företags advokat med
    den tekniska delen av försvaret. Du genomför en test av programmet, som
    till din förskräckelse visar att kunden hade rätt. Advokaten och din VD
    uppmanar dig att glömma saken, och en av dina kolleger får överta uppdraget
    att hjälpa advokaten. Vad ska du göra?
\end{quote}
%
\subsection*{Hederskoden}
Den här situationen ställer mig inför en etisk konflikt mellan lojalitet mot
min arbetsgivare och mitt ansvar som ingenjör och programmerare. I enlighet med
ingenjörens hederskodex \cite{heder} finns flera punkter som kan vägleda mig i
denna situation:
\begin{enumerate}
    \item Punkt 1 – Ingenjören bör i sin yrkesutövning känna ett personligt
          ansvar för att tekniken används på ett sätt som gagnar människa, miljö och
          samhälle.
    \item Punkt 4 – Ingenjören bör inte arbeta inom eller samverka med företag
          och organisationer av tvivelaktig karaktär eller med mål som strider mot
          personlig övertygelse. 
    \item Punkt 9 – Ingenjören bör enskilt och offentligt, i tal och skrift,
          sträva efter ett sakligt framställningssätt och undvika felaktiga,
          missvisande eller överdrivna påståenden.
    \item Punkt 10 – Ingenjören bör aktivt stödja kollegor, som råkar i
          svårigheter på grund av ett handlande i enlighet med dessa regler, samt
          enligt bästa övertygelse avstyra brott mot dem.
\end{enumerate}
Om jag skulle utgå från dessa punkter som min enda vägledning så är det första
steget att öppet och sakligt diskutera resultatet av mina tester med mina
chefer och kollegor. Argumentera om att det inte bara är oetiskt, utan också
att det kan ge långsiktiga konsekvenser såsom förlorade kunder, skadade
affärsrelationer och till och med ytterligare juridiska påföljder om kunden
senare får reda på detta.
%
\subsection*{Visselblåsning}
Om mina försök att kommunicera problemet till cheferna inte leder till någon
förändring, då tycker jag det är viktigt att dokumentera alla mina tester och
de kommunikationer jag har haft med mina chefer om situationen.  Detta kan
skydda mig om det blir en juridisk eller etisk granskning av händelsen.
Sedanefter kan jag försöka få kontakt med en högre ledning eller styrelse inom
företaget, även om det innebär att jag måste kringgå min direkta chef. Detta
kallas ibland för "intern visselblåsning". Skulle det inte heller ge något
positivt resultat får jag ytterligare saker att tänka över. Visselblåsning är
även något jag kan göra externt, alltså göra det offentligt vad företaget jag
är anställd på tänker göra, till exempel informera myndigheter eller gå till
den missnöjda kunden direkt. Det skulle jag säga är det enklaste svaret att ge,
eller kanske det svart-vita svaret.  Problemet är att det kan innebära negativa
konsekvenser för min karriär, t.ex. kan jag bli tvingad till att lämna
företaget, men det kanske är det enda etiskt försvarsbara valet om företaget
fortsätter att ignorera problemet.
%
\subsection*{Slutsats}
Att dölja testresultaten går emot flera punkter i hederskoden, inklusive min
plikt att handla ansvarsfullt (Punkt 1), respektera sanningen (Punkt 9), och
undvika samverkan med oetisk verksamhet (Punkt 4). Genom att agera för att
säkerställa att felen inte döljs, och i värsta fall rapportera problemet,
följer jag hederskodens principer om att värna om allmänhetens bästa och min
yrkesintegritet.
%
\begin{thebibliography}{99}
    \bibitem{komp} Sven Ove Hansson. \emph{Teknik och etik}.  Kompendium, 2009.
    \\
    URL: \verb|https://people.kth.se/~soh/tekniketik.pdf| \\
    Läst 2021-10-08.
    %
    \bibitem{heder} Sveriges Ingenjörer. \emph{Hederskodex}.  Kodex, 1929.  \\
    URL:
    \verb|https://www.sverigesingenjorer.se/om-forbundet/organisation/hederskodex/|
\end{thebibliography}
%
\end{document}
