\documentclass{article}
\usepackage{amsmath}

\begin{document}

För att beräkna arbetet \( W \) som utförs av gasen under en reversibel expansion där trycket ges av \( p = \frac{C}{V} \), kan vi använda följande formel för arbete i en reversibel process:

\[
W = \int_{V_i}^{V_f} p \, dV
\]

Där \( V_i \) är initialvolymen och \( V_f \) är slutvolymen. Eftersom \( p = \frac{C}{V} \), kan vi skriva om integralen som:

\[
W = \int_{V_i}^{V_f} \frac{C}{V} \, dV
\]

Integralen av \( \frac{1}{V} \) är \( \ln(V) \), så vi får:

\[
W = C \int_{V_i}^{V_f} \frac{1}{V} \, dV = C [\ln(V)]_{V_i}^{V_f}
\]

Detta ger oss:

\[
W = C (\ln(V_f) - \ln(V_i)) = C \ln\left(\frac{V_f}{V_i}\right)
\]

Nu kan vi sätta in värdena:

- \( C = 1,24 \)
- \( V_i = 2,339 \times 10^{-3} \, \text{m}^3 \)
- \( V_f = 4,61 \times 10^{-3} \, \text{m}^3 \)

\[
W = 1,24 \ln\left(\frac{4,61 \times 10^{-3}}{2,339 \times 10^{-3}}\right)
\]

vilket ger

\[
W = 1,24 \ln\left(\frac{4,61}{2,339}\right)
\]

Beräkna kvoten:

\[
\frac{4,61}{2,339} \approx 1,971
\]

Och sedan logaritmen:

\[
\ln(1,971) \approx 0,679
\]

Slutligen multiplicerar vi med konstanten \( C \):

\[
W = 1,24 \times 0,679 \approx 0,841 \, \text{J}
\]

Så arbetet som utförs av gasen under expansionen är ungefär \( 0,840 \, \text{J} \).

\end{document}